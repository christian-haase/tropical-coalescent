\documentclass[12pt]{amsart}
\usepackage[left=2.5cm,right=2.5cm, top=2.5cm, bottom=2.5cm]{geometry}
\usepackage{hyperref}
\usepackage{multirow}
\usepackage{verbatim}
\usepackage{color}
\title{Inferring gene trees from species trees}
\author{}
\date{\today}

\DeclareMathOperator{\R}{\mathbb{R}}
\DeclareMathOperator{\Var}{\mathsf{Var}}
\renewcommand{\P}{\mathbb{P}}
\newcommand{\TP}{\mathbb{TP}}
\DeclareMathOperator{\E}{\mathbb{E}}
\DeclareMathOperator{\N}{\mathbb{N}}

  \theoremstyle{plain}
\newtheorem{theorem}{Theorem}[section]
\newtheorem{lemma}[theorem]{Lemma}
\newtheorem{corollary}[theorem]{Corollary}
\newtheorem{proposition}[theorem]{Proposition}
  \theoremstyle{definition}
\newtheorem{definition}[theorem]{Definition}
\newtheorem{example}[theorem]{Example}
\newtheorem{problem}[theorem]{Problem}
\newtheorem{remark}[theorem]{Remark}

\begin{document}
\begin{abstract}
We consider statistical estimation problems for variants of the problem of Allman et al on inferring species trees from gene trees from the viewpoint of tropical geometry. 
\end{abstract}

\maketitle

\section{Overview}

The multispecies coalescent model (MSC) setup in \cite{Allman1,Allman2}  is well-studied, from both the theoretical and algorithmic perspective. For a review, see refs/reviewMSCestimation.pdf (Challenges in Species Tree Estimation Under the
Multispecies Coalescent Model, Xu and Yang). In particular, some statistics we chatted about (min pairwise coalescent times etc) have been discussed in this literature (see, for example, the section titled `Methods that use gene tree branch lengths').
Song et al (refs/song2012.pdf) applied the MSC model to data, and there are some controversies on the analysis as well as the model assumptions (refs/songReview1.pdf, refs/songReview2.pdf). 
 Thus, it is important to focus on how the tropical view can be helpful. 
\vskip12pt


\bibliographystyle{plain}
\bibliography{trees}

\end{document}
